\documentclass{beamer}

\usepackage{includes}
\usepackage{iris}

\usetheme{default}
\setbeamertemplate{navigation symbols}{
	\usebeamerfont{footline}\usebeamercolor[fg]{footline}%
	\hspace{1em}\insertframenumber
}

\begin{document}

\title{Program verification in Iris}
\author{Amin Timany}
\institute{imec-Distrinet, KU Leuven}
\date{May 2\textsuperscript{nd}, 2019 @ DRADS'19, Leuven}

\frame{\titlepage}

\begin{frame}[t]{Partial correctness}
  We want to prove partial correctness of programs, \ie, safety, of \emph{concurrent functional imperative} programs.
  \begin{itemize}
  \item Safety:
    \begin{align*}
      \SAFE{\expr}{\pred} \eqdef{} & \expr ~\text{ is safe, \ie, will not crash, and}\\
      & \text{whenever $\expr$ terminates with a value $\val$,}\\
      & \text{$\pred(\val)$ holds }
    \end{align*}
  \item Functional: firs class higher-order recursive functions:
    \begin{itemize}
    \item $\Rec f \var = \expr$
    \item Higher-order: functions can be passed around as arguments
    \end{itemize}
  \item Imperative: shared memory concurrency through references
    \begin{itemize}
    \item Memory locations: $\loc$
    \item Allocation: $\Ref(\expr)$
    \item Reading: $\deref \expr$
    \item Writing: $\expr \gets \expr'$
    \item Atomic compare-and-set operation : $\CAS(\expr_1, \expr_2, \expr_3)$
    \end{itemize}
  \end{itemize}
\end{frame}

\begin{frame}[t]{Using \Iris{}}
  We use the program logic \Iris{} :
  \begin{itemize}
  \item Hoare Triples:
    \begin{align*}
      \hoare{\prop}{\expr}{\pred}
    \end{align*}
    are defined using \Iris{}'s weakest preconditions ($\wpre{\expr}{\pred}$)
    \begin{align*}
      \hoare{\prop}{\expr}{\pred} \eqdef{} {\alert \always} (\prop {\alert \wand} \wpre{\expr}{\pred})
    \end{align*}
    \begin{itemize}
      \item $\wand$: separating implication, a.k.a., magic wand
      \item $\always$: Does not depend on \emph{ephemeral} resources, \eg, $\loc \mapsto \val$
    \end{itemize}
  \item Adequacy of \Iris{}'s weakest preconditions:
    \[ (\proves_{\mathrm{Iris}} \wpre{\expr}{\pred})  \proves_{\mathrm{Coq}} \SAFE{\expr}{\pred}\]
    And for Hoare triples:
    \[ (\proves_{\mathrm{Iris}} \hoare{\TRUE}{\expr}{\pred})  \proves_{\mathrm{Coq}} \SAFE{\expr}{\pred}\]
  \end{itemize}
\end{frame}

\begin{frame}[t]{\Iris{} rules for weakest preconditions}
  \begin{mathpar}
    \inferH{WP-val}{\pred(\val)}{\wpre{\val}{\pred}}
    \and
    \inferH{WP-bind}{\wpre{\expr}{\var.\; \wpre{\lctx[\var]}{\pred}}}{\wpre{\lctx[\expr]}{\pred}}
    \and
    \inferH{WP-lam}{\wpre{\subst{\expr}{\var}{\val}}{\pred}}{\wpre{(\Lam \var. \expr)~\val}{\pred}}
    \and
    \inferH{WP-rec}{\wpre{\subst{\expr}{\var, f}{\val, \Rec f \var = \expr}}{\pred}}{\wpre{(\Rec f \var = \expr)~\val}{\pred}}
    \and
    \inferH{WP-if-true}{\wpre{\expr_1}{\pred}}{\wpre{\If \True then \expr_1 \Else \expr_2}{\pred}}
    \and
    \inferH{WP-if-false}{\wpre{\expr_2}{\pred}}{\wpre{\If \False then \expr_1 \Else \expr_2}{\pred}}
  \end{mathpar}
\end{frame}

\begin{frame}[t]{\Iris{} rules for weakest preconditions}
  \begin{mathpar}
    \inferH{WP-load}{\loc \mapsto \val}{\wpre{\deref \loc}{\var.\; \var = \val \ast \loc \mapsto \val}}
    \and
    \inferH{WP-store}{\loc \mapsto \val}{\wpre{\loc \gets \valB}{\var.\; \var = \TT \ast \loc \mapsto \valB}}
    \and
    \inferH{WP-cas-suc}{\loc \mapsto \val}{\wpre{\CAS(\loc, \val, \valB)}{\var.\; \var = \True \ast \loc \mapsto \valB}}
    \and
    \inferH{WP-cas-fail}{\loc \mapsto \val' \and \val \neq \val'}{\wpre{\CAS(\loc, \val, \valB)}{\var.\; \var = \False \ast \loc \mapsto \val'}}
  \end{mathpar}
\end{frame}

\begin{frame}[t]{\Iris{} invariants}
  We use invariants to enforce protocols between threads:
  \[ \text{$P$ holds \emph{invariantly}}: \hspace{1em} \knowInv{\namesp}{P} \]

  Allows us to \emph{assume} $P$ before and atomic step \emph{but} we must ensure it \emph{holds} after that step

  \vspace{1em}
  Invariants are always \emph{persistent} (as opposed to ephemeral)

  \vspace{1em}
  The name $\namesp$ is used to keep track of open invariants: invariants cannot be opened twice in a \emph{nested} fashion.

\end{frame}

\end{document}